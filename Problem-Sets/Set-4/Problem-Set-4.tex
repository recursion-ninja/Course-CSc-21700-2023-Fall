\documentclass{ProblemSetCUNY}
\usepackage{bussproofs} \EnableBpAbbreviations
\usepackage{mathrsfs}
\usepackage{stix}
\usepackage{pgffor}
\usepackage{bm}
\usepackage{microtype}


\usepackage{tabstackengine}
\TABstackMath
\TABbinary

\AssignmentNumber{4}%
\CourseName{Probability and Statistics for Computer Science}
\CourseNumber{217}
\InstructorName{Alex Washburn}
\DueDateYear{2023}
\DueDateMonth{09}
\DueDateDay{25}


%\newenvironment{\ContactListing}[args]{begdef}{enddef}

\newcommand{\ProofLabelOne}[1]{\RightLabel{\ensuremath{\Parens{\mathrm{\textsc{#1}}}}\xspace}}
\newcommand{\ProofLabelTwo}[2]{\RightLabel{\ensuremath{\Parens{\mathrm{\textsc{#1}}}\Parens{\mathrm{#2}}}\xspace}}

\newcommand{\LogicCase}[1]{\item \textbf{Case}~#1:\newline}

\newcommand{\DerivationFormat}[2]{\hfill\qquad#2\textsc{#1}}
\newcommand{\Derivation}[2]{\DerivationFormat{#1}{#2~\ensuremath{\ruledelayed}~}}

\newcommand{\AxiomCL}[1]{\DerivationFormat{Axiom~#1}{}}
\newcommand{\Factivity}{\DerivationFormat{Factivity}{}}
\newcommand{\Given}{\DerivationFormat{Hypothesis}{}}
\newcommand{\Contrapositive}{\DerivationFormat{Contrapositive}{}}
\newcommand{\DoubleNegation}{\DerivationFormat{Double Negation}{}}
\newcommand{\DeMorgans}{\DerivationFormat{DeMorgans}{}}
\newcommand{\Necitation}[1]{\Derivation{Necitation}{#1}}
\newcommand{\Distribution}[1]{\Derivation{Distribution}{#1}}
\newcommand{\Reflexivity}[1]{\Derivation{Reflexivity}{#1}}
\newcommand{\ConDef}[1]{\Derivation{Connective Definition}{#1}}
\newcommand{\Contradiction}[1]{\ensuremath{\neg \phi \land \phi}\Derivation{Contradiction}{\ensuremath{\phi = #1}}}
\newcommand{\DedThm}[1]{\Derivation{Deduction Theorem}{#1}}
\newcommand{\ModPon}[2]{\Derivation{Modus Ponens}{#1, #2}}
\newcommand{\Syllogism}[2]{\Derivation{Syllogism}{#1, #2}}
\newcommand{\Theorem}[2]{\DerivationFormat{Theorem~--~Lecture~#1,~Page~#2}{}}

\newcommand{\R}[2]{\ensuremath{\World{#1}\mathrel{R}\World{#2}}\xspace}
\newcommand{\Tr}[2]{\ensuremath{\tau\Parens{#1,\,#2}\xspace}}
\newcommand{\T}[1]{\ensuremath{\tau\Parens{#1}\xspace}}
\newcommand{\Prob}[1]{\ensuremath{P\Parens{#1}}\xspace}
\newcommand{\ProbGiven}[2]{\ensuremath{P\Parens{#1\,|\,#2}}\xspace}
\newcommand{\Event}[1]{\ensuremath{\bm{\mathsf{#1}}}\xspace}
\newcommand{\RandVar}[1]{\ensuremath{\mathbf{#1}}\xspace}
\newcommand{\DistMap}[2]{~#1&\mapsto~~#2\\}
\newcommand{\DistCondition}[2]{~#1&\text{\normalfont iff}~~#2\\}

\newcommand{\Hint}[1]{\textit{\textls{Hint:}~#1}}

\newcommand{\One}{\ensuremath{\textbf{1}}\xspace}

\newcommand{\Heads}{\ensuremath{\mathtt{H}}\xspace}
\newcommand{\Tails}{\ensuremath{\mathtt{T}}\xspace}

\newcommand{\Die}[1]{\ensuremath{\mathbf{D}_{#1}}\xspace}
\newcommand{\Above}[1]{\ensuremath{\overline{#1}}\xspace}
\newcommand{\Below}[1]{\ensuremath{\underline{#1}}\xspace}
\newcommand{\Num}[1]{\ensuremath{\mathbb{#1}}\xspace}
\newcommand{\NumA}[1]{\ensuremath{\overline{\Num{#1}}}\xspace}
\newcommand{\NumB}[1]{\ensuremath{\underline{\Num{#1}}}\xspace}


\newcommand{\Deck}{\ensuremath{\bm{\mathcal{D}}}\xspace}
\newcommand{\CardSuits}{\ensuremath{\bm{\mathcal{S}}}\xspace}
\newcommand{\CardRanks}{\ensuremath{\bm{\mathcal{R}}}\xspace}
\newcommand{\CardRank}[1]{\ensuremath{\text{\textsf{\textbf{#1}}}}\xspace}

\newcommand{\SizeOf}[1]{\ensuremath{\left|#1\right| }\xspace}

\newcommand{\Countermodel}[4]{%
\textbf{Countermodel} $\mathcal{K}$ in #1:%
\begin{itemize}%
\item \ensuremath{W = \SetNote{\World{0}%
\foreach \n in {1,...,#2}{,~\World{\n}}%
}}%
\item \ensuremath{\mathrel{R}~=~#3}%
\item #4%
\end{itemize}%
}

\begin{document}
\CoverPage%

\newcommand{\Bullet}[3]{\hspace*{4mm}\ensuremath{\Parens{\Event{#1}}}:\hspace*{2mm}#2\[#3\]\\[5mm]}
\newcommand{\BulletLine}[3]{\hspace*{4mm}\ensuremath{\Parens{\Event{#1}}}:\hspace*{2mm}#2\hfill\ensuremath{#3}\\[5mm]}

\newcommand{\Slot}{\ensuremath{\_\_\_}\xspace}

\Problem{%
Alice has a ``deck'' \Deck, which is a collection of $52$ playing cards.
Each ``card'' is a unique pairing of a suit and rank sets \CardSuits and \CardRanks, respectively.
\[\CardSuits = 
\SetNote{
\clubsuit,\,
\vardiamondsuit,\,
\spadesuit,\, 
\varheartsuit
}\]
\[\CardRanks = \SetNote{
\CardRank{2},\,
\CardRank{3},\,
\CardRank{4},\,
\CardRank{5},\,
\CardRank{6},\,
\CardRank{7},\,
\CardRank{8},\,
\CardRank{9},\,
\CardRank{10},\,
\CardRank{J},\,
\CardRank{Q},\,
\CardRank{K},\,
\CardRank{A}
}\]
\[\Omega = \Deck = \CardSuits \times \CardRanks = \SetNote{ (s,\,r)\;|\; s \in \CardSuits,\, r \in \CardRanks}\]
Alice deals out a``hand,'' which is an \emph{unordered} subset of \emph{five} cards from \Deck, without replacement; i.e. since each card is unique pair of suit and rank, a hand can never contain two cards with both the same suit and rank.
Each possible hand dealt from \Deck occurs with equal probability.\\[5mm]
\emph{How many different possible hands can be deal from \Deck?}
}


\begin{minipage}[t]{.49\textwidth}
There are $5$ cards to be dealt out from $52$ unique options.
\[
\Slot\qquad\Slot\qquad\Slot\qquad\Slot\qquad\Slot
\]
There are $52$ possibilities for the 1\textsuperscript{st} card:
\[
\underline{~~52~~}\qquad\Slot\qquad\Slot\qquad\Slot\qquad\Slot
\]
There are now $51$ cards left in \Deck,\\so there are $51$ possibilities for the 2\textsuperscript{nd} card:
\[
\underline{~~52~~}~\times~\underline{~~51~~}\qquad\Slot\qquad\Slot\qquad\Slot
\]
\Deck now only has $50$ cards remaining,\\and the pattern continues for the 3\textsuperscript{rd}, 4\textsuperscript{th}, and 5\textsuperscript{th} cards:
\[
\underline{~~52~~}~\times~\underline{~~51~~}~\times~\underline{~~50~~}\qquad\Slot\qquad\Slot
\]
\[
\underline{~~52~~}~\times~\underline{~~51~~}~\times~\underline{~~50~~}~\times~\underline{~~49~~}\qquad\Slot
\]
\[
\underline{~~52~~}~\times~\underline{~~51~~}~\times~\underline{~~50~~}~\times~\underline{~~49~~}~\times~\underline{~~48~~}
\]
However, this result considers all the possible \emph{orderings} of $5$ cards dealt from the deck \Deck with $52$ cards, but hands are \emph{unordered}.
To find the number of \emph{unordered} hands, one must divide the product above by the number of possible permutations for $5$ cards, which is $\Parens{5!}$:
\[
\frac{52\times51\times50\times49\times48}{5!} = \binom{52}{5}
\]
\end{minipage}
\begin{minipage}[t]{.49\textwidth}
\hspace*{5mm}The answer is:
\begin{align*}
\binom{52}{5} &= \frac{52!}{\Parens{52 - 5}! \times 5!}\\
&= \frac{52!}{47! \times 5!}\\
&= \frac{52\times51\times50\times49\times48\times47!}{47! \times 5!}\\
&= \frac{52\times51\times50\times49\times48}{5\times4!}\\
&= \frac{52\times51\times10\times49\times48}{4!}\\
&= 52\times51\times10\times49\times2\\
&= 2,598,960
\end{align*}
\end{minipage}

\newcommand{\Digits}{\ensuremath{\mathbb{D}}\xspace}
\Problem{%
Alice deals out a hand and notices that the rank of each card is a digit, i.e. each rank is an element of the following subset of \CardRanks:
\[ \SetNote{
\CardRank{2},\,
\CardRank{3},\,
\CardRank{4},\,
\CardRank{5},\,
\CardRank{6},\,
\CardRank{7},\,
\CardRank{8},\,
\CardRank{9},\,
\CardRank{10}
} = \Digits \subset \CardRanks
\]
\emph{What is the probability of Alice dealing such a hand from \Deck?}
\[\Prob{\textsc{Rank}(H_i) \in \Digits;~\;\forall i \in \NumericRange{1}{5}}\]
}
\begin{minipage}[t]{.48\textwidth}
The number of cards we want to consider is a proper subset of the full \Deck.
\[
\SizeOf{\Digits} \times \SizeOf{\CardSuits} = 9 \times 4 = 36
\]
There are $5$ cards to be dealt out from the $36$ unique options.
\[
\Slot\qquad\Slot\qquad\Slot\qquad\Slot\qquad\Slot
\]
There are $36$ possibilities for the 1\textsuperscript{st} card:
\[
\underline{~~36~~}\qquad\Slot\qquad\Slot\qquad\Slot\qquad\Slot
\]
There are now $35$ \emph{digit} cards left in \Deck,\\so there are $35$ possibilities for the 2\textsuperscript{nd} card:
\[
\underline{~~36~~}~\times~\underline{~~35~~}\qquad\Slot\qquad\Slot\qquad\Slot
\]
\Deck now only has $34$ \emph{digit} cards remaining,\\the pattern continues for the 3\textsuperscript{rd}, 4\textsuperscript{th}, and 5\textsuperscript{th} cards:
\[
\underline{~~36~~}~\times~\underline{~~35~~}~\times~\underline{~~34~~}\qquad\Slot\qquad\Slot
\]
\[
\underline{~~36~~}~\times~\underline{~~35~~}~\times~\underline{~~34~~}~\times~\underline{~~33~~}\qquad\Slot
\]
\[
\underline{~~36~~}~\times~\underline{~~35~~}~\times~\underline{~~34~~}~\times~\underline{~~33~~}~\times~\underline{~~32~~}
\]
Similar to Problem 1, this result considers all the possible \emph{orderings} of $5$ \emph{digit} cards dealt from \Deck, but hands are \emph{unordered}.
To find the number of \emph{unordered} hands, one must divide the product above by the number of possible permutations for $5$ cards, which is $\Parens{5!}$:
\[
\frac{36\times35\times34\times33\times32}{5!} = \binom{36}{5}
\]
\begin{align*}
\text{\# All digit hands}& = \binom{36}{5}\\
&= \frac{36!}{\Parens{36 - 5}! \times 5!}\\
&= \frac{36!}{31! \times 5!}\\
&= \frac{36\times35\times34\times33\times32\times31!}{31! \times 5!}\\
&= \frac{36\times35\times34\times33\times32}{5\times4!}\\
&= \frac{36\times7\times34\times33\times32}{4\times3!}\\
&= \frac{36\times7\times34\times33\times8}{3!}\\
&= 6\times7\times34\times33\times8\\
&= 376,992
\end{align*}~\vfill\end{minipage}
\begin{minipage}[t]{.48\textwidth}
\hspace*{5mm}With the number of ``all digit hands'' known, we can compute the probability of such a hand being dealt:
\begin{align*}
\Prob{\textsc{Rank}(H_i) \in \Digits;~\;\forall i \in \NumericRange{1}{5}} &= \frac{\binom{36}{5}}{\binom{52}{5}}\\
&= \frac{376,992}{2,598,960}\\
&= \frac{66}{455}\approx 0.145
\end{align*}
\end{minipage}

\Problem{
Bob flips a fair coin $100$ times.
Bob observes that after $100$ coin flips, he observed $52$ heads (\Heads) and $48$ heads (\Tails).\\[5mm]
\emph{What is \Prob{52~\Heads}; i.e the probability of 52 \Heads occurring in any order?}
}

Let $\Prob{\Heads} = p$ and $\Prob{\Tails} = 1 - p$.
By using the binomial formula, we arrive at:

\begin{align*}
\Prob{52~\Heads} &= \binom{100}{52}\times p^{52} \times \Parens{1-p}^{48}
\end{align*}

Alternatively, if we assume that $p = \frac{1}{2}$, then we can take the ratio of possibilities:

\begin{align*}
\Prob{52~\Heads} &= \frac{\binom{100}{52}}{2^{100}}
\end{align*}

Simplifying the expression (not required) yields the following:

\begin{align*}
\frac{\binom{100}{52}}{2^{100}} &= \frac{100!}{52! \times 2^{100} \times \Parens{100 - 52}!}\\
&= \frac{\Parens{2 \times 50}!}{52 \times 51 \times 50! \times 2^{50} \times 2^{50} \times 48!}\\
&= \frac{\prod_{i=1}^{50} \Parens{2i-1}}{52 \times 51 \times 2^{50}\times 48!}\\
&= \frac{\prod_{i=25}^{50} \Parens{2i-1}}{52 \times 51 \times 2^{50}\times \prod_{i=2}^{24}\Parens{2i}}\\
&= \frac{51 \times 49 \times \prod_{i=27}^{50} \Parens{2i-1}}{52 \times 51 \times 2^{50} \times \prod_{i=1}^{24}\Parens{2i}}\\
&= \frac{51 \times 50 \times 49 \times \prod_{i=27}^{50} \Parens{2i-1}}{52 \times 51 \times 50 \times 2^{50} \times \prod_{i=2}^{24}\Parens{2i}}\\
&= \frac{5^2 \times 7^2 \times \prod_{i=27}^{50} \Parens{2i-1}}{52 \times 50 \times 2^{49} \times \prod_{i=1}^{24}\Parens{2i}}\\
&= \frac{5^2 \times 7^2 \times \prod_{i=27}^{50} \Parens{2i-1}}{2^{49} \times \prod_{i=1}^{26}\Parens{2i}\phantom{\,}}\\
&\approx 0.0735 \approx \frac{47 \times 2^{94}}{\phantom{4}~5\times 2^{101}}
\end{align*}


\StartExtraCredit
\Problem{%
Which is the most preferable from these mutually exclusive options?\\[5mm]%
\setstretch{1.7}
\hspace*{1cm}$\tabbedLongunderstack[l]{
\,\heartsuit\,\quad\text{Pursue your passion}\\
\;\$\:\quad\text{Take the money}%
}$%
}



\end{document}
