\documentclass{ProblemSetCUNY}
\usepackage{bussproofs} \EnableBpAbbreviations
\usepackage{mathrsfs}
\usepackage{stix}
\usepackage{pgffor}
\usepackage{booktabs}
\usepackage{nicematrix}

\renewcommand{\arraystretch}{1.2}

\usepackage{tabstackengine}
\TABstackMath
\TABbinary

\AssignmentNumber{1}%
\CourseName{Probability and Statistics for Computer Science}
\CourseNumber{217}
\InstructorName{Alex Washburn}
\DueDateYear{2023}
\DueDateMonth{09}
\DueDateDay{05}


%\newenvironment{\ContactListing}[args]{begdef}{enddef}

\newcommand{\ProofLabelOne}[1]{\RightLabel{\ensuremath{\Parens{\mathrm{\textsc{#1}}}}\xspace}}
\newcommand{\ProofLabelTwo}[2]{\RightLabel{\ensuremath{\Parens{\mathrm{\textsc{#1}}}\Parens{\mathrm{#2}}}\xspace}}

\newcommand{\LogicCase}[1]{\item \textbf{Case}~#1:\newline}

\newcommand{\DerivationFormat}[2]{\hfill\qquad#2\textsc{#1}}
\newcommand{\Derivation}[2]{\DerivationFormat{#1}{#2~\ensuremath{\ruledelayed}~}}

\newcommand{\AxiomCL}[1]{\DerivationFormat{Axiom~#1}{}}
\newcommand{\Factivity}{\DerivationFormat{Factivity}{}}
\newcommand{\Given}{\DerivationFormat{Hypothesis}{}}
\newcommand{\Contrapositive}{\DerivationFormat{Contrapositive}{}}
\newcommand{\DoubleNegation}{\DerivationFormat{Double Negation}{}}
\newcommand{\DeMorgans}{\DerivationFormat{DeMorgans}{}}
\newcommand{\Necitation}[1]{\Derivation{Necitation}{#1}}
\newcommand{\Distribution}[1]{\Derivation{Distribution}{#1}}
\newcommand{\Reflexivity}[1]{\Derivation{Reflexivity}{#1}}
\newcommand{\ConDef}[1]{\Derivation{Connective Definition}{#1}}
\newcommand{\Contradiction}[1]{\ensuremath{\neg \phi \land \phi}\Derivation{Contradiction}{\ensuremath{\phi = #1}}}
\newcommand{\DedThm}[1]{\Derivation{Deduction Theorem}{#1}}
\newcommand{\ModPon}[2]{\Derivation{Modus Ponens}{#1, #2}}
\newcommand{\Syllogism}[2]{\Derivation{Syllogism}{#1, #2}}
\newcommand{\Theorem}[2]{\DerivationFormat{Theorem~--~Lecture~#1,~Page~#2}{}}

\newcommand{\R}[2]{\ensuremath{\World{#1}\mathrel{R}\World{#2}}\xspace}
\newcommand{\Tr}[2]{\ensuremath{\tau\Parens{#1,\,#2}\xspace}}
\newcommand{\T}[1]{\ensuremath{\tau\Parens{#1}\xspace}}
\newcommand{\Prob}[1]{\ensuremath{P\Parens{#1}}\xspace}
\newcommand{\RandVar}[1]{\ensuremath{\mathbf{#1}}\xspace}
\newcommand{\DistMap}[2]{~#1&\mapsto~~#2\\}
\newcommand{\DistCondition}[2]{~#1&\text{\normalfont iff}~~#2\\}


\newcommand{\Heads}{\ensuremath{\mathtt{H}}\xspace}
\newcommand{\Tails}{\ensuremath{\mathtt{T}}\xspace}


\newcommand{\One}{\ensuremath{\textbf{1}}\xspace}

\newcommand{\Countermodel}[4]{%
\textbf{Countermodel} $\mathcal{K}$ in #1:%
\begin{itemize}%
\item \ensuremath{W = \SetNote{\World{0}%
\foreach \n in {1,...,#2}{,~\World{\n}}%
}}%
\item \ensuremath{\mathrel{R}~=~#3}%
\item #4%
\end{itemize}%
}

\begin{document}
\CoverPage%

\setstretch{1.2}

\Problem{%
You flip a fair coin 4 times.\\
All flips are mutually independent and have outcomes $\Omega = \SetNote{\Heads,\,\Tails}$.\\
Determine the probability of the events below.\\[5mm]
\textit{Hint: Draw a table enumerating the all possible outcomes of 4 flips.}
}

The full space of possible observations from four coin flips is $\Omega \times \Omega \times \Omega \times \Omega = \Omega^{4}$ which is enumerated below.
The total number of observations in $\Omega^{4}$ is $\left\lvert\Omega^{4}\right\rvert = 16$.
Each observation is equally probable, hence each observation $\RandVar{C}$ in the table below has probability $\Prob{\RandVar{C}} = \frac{1}{16}$. 

\begin{tabular}{@{}r|llll@{}}
\toprule
  & $1^{st}$ & $2^{nd}$ & $3^{rd}$ & $4^{th}$ \\
\midrule
1 & \Tails & \Tails & \Tails & \Tails \\
2 & \Tails & \Tails & \Tails & \Heads\\
3 & \Tails & \Tails & \Heads & \Tails \\
4 & \Tails & \Tails & \Heads& \Heads\\
5 & \Tails & \Heads & \Tails & \Tails \\
6 & \Tails & \Heads & \Tails & \Heads\\
7 & \Tails & \Heads & \Heads & \Tails \\
8 & \Tails & \Heads & \Heads& \Heads\\
9 & \Heads & \Tails & \Tails & \Tails \\
10 & \Heads & \Tails & \Tails & \Heads\\
11 & \Heads & \Tails & \Heads & \Tails \\
12 & \Heads & \Tails & \Heads& \Heads\\
13 & \Heads & \Heads & \Tails & \Tails \\
14 & \Heads & \Heads & \Tails & \Heads\\
15 & \Heads & \Heads & \Heads & \Tails \\
16 & \Heads & \Heads & \Heads& \Heads\\
\bottomrule
\end{tabular}

\SubProblem{Four heads: \Parens{\Heads,\,\Heads,\,\Heads,\,\Heads}}

There is only one occurrence of four heads \Parens{\Heads,\,\Heads,\,\Heads,\,\Heads}. Hence the answer is: 

\[
\frac{1}{\left\lvert\Omega^{4}\right\rvert} = \frac{1}{16}
\]


\SubProblem{The sequence: \Parens{\Heads,\,\Tails,\,\Heads,\,\Tails}}

There is only one occurrence of the sequence \Parens{\Heads,\,\Tails,\,\Heads,\,\Tails}. Hence the answer is: 

\[
\frac{1}{\left\lvert\Omega^{4}\right\rvert} = \frac{1}{16}
\]


\SubProblem{Any sequence with $3~\Heads$ and $1~\Tails$}

There are four occurrences of sequences satisfying $3~\Heads$ and $1~\Tails$; specifically rows 8, 12, 14, and 15.  Hence the answer is: 

\[
\frac{4}{\left\lvert\Omega^{4}\right\rvert} = \frac{4}{16} = \frac{1}{4}
\]


\SubProblem{Any sequence where the number of \Heads is greater than or equal to \Tails}

There are eleven occurrences of sequences satisfying the number of \Heads is greater than or equal to \Tails.
We can arrive at this by counting the number of rolls with two \Heads, three \Heads, and four \Heads, then taking thier summation (since they are mutually independent).

\begin{itemize}
\item \parbox{10mm}{Two}$\Heads$:~$\quad\binom{4}{2} \times  \frac{1}{16} \;=\; 6 \times \frac{1}{16} \;=\; \frac{6}{16}$
\item \parbox{10mm}{Three}$\Heads$:~$\quad\binom{4}{3} \times  \frac{1}{16} \;=\; 4 \times \frac{1}{16} \;=\; \frac{4}{16}$
\item \parbox{10mm}{Four}$\Heads$:~$\quad\binom{4}{4} \times  \frac{1}{16} \;=\; 1 \times \frac{1}{16} \;=\; \frac{1}{16}$
\end{itemize}

\[
\frac{6}{16} +  \frac{4}{16} + \frac{1}{16} = \frac{11}{16}
\]


\newcommand{\Die}[1]{\ensuremath{\mathbf{D}_{#1}}\xspace}
\newcommand{\Above}[1]{\ensuremath{\overline{#1}}\xspace}
\newcommand{\Below}[1]{\ensuremath{\underline{#1}}\xspace}
\newcommand{\Num}[1]{\ensuremath{\mathbb{#1}}\xspace}
\newcommand{\NumA}[1]{\ensuremath{\overline{\Num{#1}}}\xspace}
\newcommand{\NumB}[1]{\ensuremath{\underline{\Num{#1}}}\xspace}



\Problem{%
You throw two fair, 6-faced dice, one green and one blue, called \Die{1} \& \Die{2}.\\
The result of each die is independent of the other.\\
$\Die{1} \text{ has outcomes } \Omega_{\Die{1}} = \SetNote{\NumA{1},\,\NumA{2},\,\NumA{3},\,\NumA{4},\,\NumA{5},\,\NumA{6}}$.\\
$\Die{2} \text{ has outcomes } \Omega_{\Die{2}} = \SetNote{\NumB{1},\,\NumB{2},\,\NumB{3},\,\NumB{4},\,\NumB{5},\,\NumB{6}}$.\\
Determine the probability of the events below.\\[5mm]
\textit{Hint: Draw a $6 \times 6$ grid enumerating the all possible rolls of the dice.}
}

$\begin{NiceArray}{*{7}{c}}[hvlines] 
\diagbox{\Die{1}}{\Die{2}} & \NumA{1} & \NumA{2} & \NumA{3} & \NumA{4} & \NumA{5} & \NumA{6}\\
~~~~\NumB{1}~~~~ & \Parens{\NumA{1},\NumB{1}} & \Parens{\NumA{2},\NumB{1}} & \Parens{\NumA{3},\NumB{1}} & \Parens{\NumA{4},\NumB{1}} & \Parens{\NumA{5},\NumB{1}} & \Parens{\NumA{6},\NumB{1}} \\
\NumB{2} & \Parens{\NumA{1},\NumB{2}} & \Parens{\NumA{2},\NumB{2}} & \Parens{\NumA{3},\NumB{2}} & \Parens{\NumA{4},\NumB{2}} & \Parens{\NumA{5},\NumB{2}} & \Parens{\NumA{6},\NumB{2}} \\
\NumB{3} & \Parens{\NumA{1},\NumB{3}} & \Parens{\NumA{2},\NumB{3}} & \Parens{\NumA{3},\NumB{3}} & \Parens{\NumA{4},\NumB{3}} & \Parens{\NumA{5},\NumB{3}} & \Parens{\NumA{6},\NumB{3}} \\
\NumB{4} & \Parens{\NumA{1},\NumB{4}} & \Parens{\NumA{2},\NumB{4}} & \Parens{\NumA{3},\NumB{4}} & \Parens{\NumA{4},\NumB{4}} & \Parens{\NumA{5},\NumB{4}} & \Parens{\NumA{6},\NumB{4}} \\
\NumB{5} & \Parens{\NumA{1},\NumB{5}} & \Parens{\NumA{2},\NumB{5}} & \Parens{\NumA{3},\NumB{5}} & \Parens{\NumA{4},\NumB{5}} & \Parens{\NumA{5},\NumB{5}} & \Parens{\NumA{6},\NumB{5}} \\
\NumB{6} & \Parens{\NumA{1},\NumB{6}} & \Parens{\NumA{2},\NumB{6}} & \Parens{\NumA{3},\NumB{6}} & \Parens{\NumA{4},\NumB{6}} & \Parens{\NumA{5},\NumB{6}} & \Parens{\NumA{6},\NumB{6}} \\
\end{NiceArray}$


\SubProblem{At least one 3 occurs; $\Prob{\Die{1} = \NumA{3}}$ or $\Prob{\Die{2} = \NumB{3}}$}

\[
\Prob{\Die{1} = \NumA{3}} + \Prob{\Die{2} = \NumB{3}} - \Prob{\Die{1} = \NumA{3}\;\land\;\Die{2} = \NumB{3}} = \frac{6}{36} + \frac{6}{36} - \frac{1}{36} = \frac{11}{36} 
\]


\SubProblem{The larger of the two dice is greater than 4; $\Prob{\max\Parens{\Die{1},\,\Die{2}} > \Num{4}}$}

\[
\Prob{\Die{1} > \NumA{4}} + \Prob{\Die{2} > \NumB{4}} - \Prob{\Die{1} > \NumA{4}\;\land\;\Die{2} > \NumB{4}} = \frac{12}{36} + \frac{12}{36} - \frac{4}{36} = \frac{20}{36} = \frac{5}{9}
\]


\SubProblem{The faces of the dice match; $\Prob{\Die{1} = \Die{2}}$}

\begin{align*}
\Prob{\Die{1} = \Die{2}} &= \sum\limits_{x \in \Omega} \Prob{\Die{1} = \Die{2} = x}\\
&= \Prob{\Die{1} = \Die{2} = \Num{1}} + \Prob{\Die{1} = \Die{2} = \Num{2}} + \Prob{\Die{1} = \Die{2} = \Num{3}} \\
&+\,\Prob{\Die{1} = \Die{2} = \Num{4}} + \Prob{\Die{1} = \Die{2} = \Num{5}} + \Prob{\Die{1} = \Die{2} = \Num{6}} \\
&= \frac{1}{36} + \frac{1}{36} + \frac{1}{36} + \frac{1}{36} + \frac{1}{36} + \frac{1}{36} \\
&= \frac{6}{36} \\
&= \frac{1}{6}
\end{align*}


\SubProblem{The sum of the dice is 7; $\Prob{\Die{1} + \Die{2} = \Num{7}}$}

\begin{align*}
\Prob{\Die{1} + \Die{2} = \Num{7}} &= \Prob{\Die{1} = \NumA{1}\;\land\;\Die{2} = \NumB{6}} + \Prob{\Die{1} = \NumA{2}\;\land\;\Die{2} = \NumB{5}} + \Prob{\Die{1} = \NumA{3}\;\land\;\Die{2} = \NumB{4}} \\
&+\,\Prob{\Die{1} = \NumA{4}\;\land\;\Die{2} = \NumB{3}} + \Prob{\Die{1} = \NumA{5}\;\land\;\Die{2} = \NumB{2}} + \Prob{\Die{1} = \NumA{6}\;\land\;\Die{2} = \NumB{1}} \\
&= \frac{1}{36} + \frac{1}{36} + \frac{1}{36} + \frac{1}{36} + \frac{1}{36} + \frac{1}{36} \\
&= \frac{6}{36} \\
&= \frac{1}{6}
\end{align*}

\Problem{%
You the same \Die{1} and \Die{2} from Problem 2. However, this time you are using \Die{1} and \Die{2} as the dice to play the game of Monopoly. In the game of Monopoly, if you ``roll doubles,'' meaning that the faces of the dice are the same number, then you get to roll the dice again. You can roll the dice a maximum of three times, if you happen to ``roll doubles'' on both the first and second rolls.
The number of spaces you move is the sum of all dice rolls.\\[5mm]
What is the probability that you move \emph{at least} 12 spaces?\\[5mm]
\textit{Hint: Draw a $6 \times 6$ grid enumerating the all possible rolls of the dice. Then sub-divide the cells where you ``roll doubles'' into another, smaller $6 \times 6$ grid within that cell. Each cell, no matter how small, is one possible outcome.}
}

\newcommand{\ReRoll}[1]{\ensuremath{\left\lVert\;#1\;\right\rVert}}
\newcommand{\Roll}[2]{\ensuremath{\textbf{Roll}_{\textbf{#1}}\Parens{#2}}\xspace}

The diagonal contains the \emph{six} possibilities for another roll to occur:

$\begin{NiceArray}{*{7}{c}}[hvlines] 
\diagbox{\Die{1}}{\Die{2}} & \NumA{1} & \NumA{2} & \NumA{3} & \NumA{4} & \NumA{5} & \NumA{6}\\
~~~~\NumB{1}~~~~ & \ReRoll{\NumA{1},\NumB{1}} & \Parens{\NumA{2},\NumB{1}} & \Parens{\NumA{3},\NumB{1}} & \Parens{\NumA{4},\NumB{1}} & \Parens{\NumA{5},\NumB{1}} & \Parens{\NumA{6},\NumB{1}} \\
\NumB{2} & \Parens{\NumA{1},\NumB{2}} & \ReRoll{\NumA{2},\NumB{2}} & \Parens{\NumA{3},\NumB{2}} & \Parens{\NumA{4},\NumB{2}} & \Parens{\NumA{5},\NumB{2}} & \Parens{\NumA{6},\NumB{2}} \\
\NumB{3} & \Parens{\NumA{1},\NumB{3}} & \Parens{\NumA{2},\NumB{3}} & \ReRoll{\NumA{3},\NumB{3}} & \Parens{\NumA{4},\NumB{3}} & \Parens{\NumA{5},\NumB{3}} & \Parens{\NumA{6},\NumB{3}} \\
\NumB{4} & \Parens{\NumA{1},\NumB{4}} & \Parens{\NumA{2},\NumB{4}} & \Parens{\NumA{3},\NumB{4}} & \ReRoll{\NumA{4},\NumB{4}} & \Parens{\NumA{5},\NumB{4}} & \Parens{\NumA{6},\NumB{4}} \\
\NumB{5} & \Parens{\NumA{1},\NumB{5}} & \Parens{\NumA{2},\NumB{5}} & \Parens{\NumA{3},\NumB{5}} & \Parens{\NumA{4},\NumB{5}} & \ReRoll{\NumA{5},\NumB{5}} & \Parens{\NumA{6},\NumB{5}} \\
\NumB{6} & \Parens{\NumA{1},\NumB{6}} & \Parens{\NumA{2},\NumB{6}} & \Parens{\NumA{3},\NumB{6}} & \Parens{\NumA{4},\NumB{6}} & \Parens{\NumA{5},\NumB{6}} & \ReRoll{\NumA{6},\NumB{6}} \\
\end{NiceArray}$

\[
30 + 6 \times \Parens{30 + 6 \times \Parens{30 + 6} } = 1506 = \left|\Omega\right|
\]

\begin{itemize}
\item When \Roll{1}{\Die{1} \not= \Die{2}} then \Prob{\textbf{spaces} \ge 12} = $\frac{0}{1506}$
\item When \Roll{1}{\Die{1} = \Die{2} = \Num{6}} then \Prob{\textbf{spaces} \ge 12} = $\frac{30 + 6 \times 36}{1506} = \frac{246}{1506}$\\
Because the first roll is $6 + 6 \ge 12$, and all subsequent rolls are monotonically increasing the number of spaces.
\item When \Roll{1}{\Die{1} = \Die{2} = \Num{5}} then \Prob{\textbf{spaces} \ge 12} = $\frac{30 + 6 \times 36}{1506} = \frac{246}{1506}$\\
Because the first roll plus the minimum possible second roll is $5 + 5 + 1 + 1 \ge 12$, and all subsequent rolls are monotonically increasing the number of spaces.
\item When \Roll{1}{\Die{1} = \Die{2} = \Num{4}} then \Prob{\textbf{spaces} \ge 12} = $\frac{28 + 6 \times 36}{1506} = \frac{244}{1506}$\\
Because the first roll is $4 + 4 = 8$, there are \emph{two} possible second rolls which do not result in a another roll and also do not have a cumulative sum $\ge 12$, notably \Parens{\NumA{1}, \NumB{2}} and \Parens{\NumA{2}, \NumB{1}}. All possible doubles will yield a result $\ge 12$, either immediately on the second roll or on the third and final roll.
\item When \Roll{1}{\Die{1} = \Die{2} = \Num{3}} then \Prob{\textbf{spaces} \ge 12} = $\frac{33}{1506} + \frac{36}{1506} + \frac{22 + 4 \times 36}{1506} = \frac{235}{1506}$\\
Because the first roll is $3 + 3 = 6$, there are \emph{eight} possible second rolls which do not result in a another roll and also do not have a cumulative sum $\ge 12$, notably:\\
\Parens{\NumA{1}, \NumB{2}}, \Parens{\NumA{1}, \NumB{3}}, \Parens{\NumA{1}, \NumB{4}}, \Parens{\NumA{2}, \NumB{1}}, \Parens{\NumA{2}, \NumB{3}}, \Parens{\NumA{3}, \NumB{1}}, \Parens{\NumA{3}, \NumB{2}} and \Parens{\NumA{4},  \NumB{1}}.\\
There are \emph{four} possible second rolls which are doubles and do immediately yield a result $\ge 12$, resulting in a partial answer of $\langle?\rangle + \frac{22 + 4 \times 36}{1506}$.\\
There are \emph{two} possible second rolls which are doubles and \emph{do not} immediately yield a result $\ge 12$ which we must consider to determine that $\langle?\rangle = \frac{33}{1506} + \frac{36}{1506}$:
\begin{itemize}
\item When \Roll{2}{\Die{1} = \Die{2} = \Num{1}} then \Prob{\textbf{spaces} \ge 12} = $\frac{33}{1506}$\\
Because the first roll plus the second roll is $3 + 3 + 1 + 1 = 8$, there are \emph{three} possible third rolls which do not yield a number of spaces $\ge 12$, notably  \Parens{\NumA{1}, \NumB{1}}, \Parens{\NumA{1}, \NumB{2}} and \Parens{\NumA{2}, \NumB{1}}.
\item When \Roll{2}{\Die{1} = \Die{2} = \Num{2}} then \Prob{\textbf{spaces} \ge 12} = $\frac{36}{1506}$\\
Because the first roll plus the second roll plus the minimum possible third roll is $3 + 3 + 2 + 2 + 1 + 1 \ge 12$
\end{itemize}

\item When \Roll{1}{\Die{1} = \Die{2} = \Num{2}} then \Prob{\textbf{spaces} \ge 12} = $ \frac{26}{1506} + \frac{33}{1506} + \frac{36}{1506} + \frac{12 + 3 \times 36}{1506} = \frac{215}{1506}$\\
Because the first roll is $2 + 2 = 4$, there are \emph{eighteen} possible second rolls which do not result in a another roll and also do not have a cumulative sum $\ge 12$, notably:\\
\Parens{\NumA{1}, \NumB{2}}, \Parens{\NumA{1}, \NumB{3}}, \Parens{\NumA{1}, \NumB{4}},  \Parens{\NumA{1}, \NumB{5}}, \Parens{\NumA{1}, \NumB{6}}, \Parens{\NumA{2}, \NumB{1}}, \Parens{\NumA{2}, \NumB{3}}, \Parens{\NumA{2}, \NumB{4}}, \Parens{\NumA{2}, \NumB{5}},\\
\Parens{\NumA{3}, \NumB{1}}, \Parens{\NumA{3}, \NumB{2}}, \Parens{\NumA{3}, \NumB{4}}, \Parens{\NumA{4}, \NumB{1}}, \Parens{\NumA{4}, \NumB{2}}, \Parens{\NumA{4}, \NumB{3}}, \Parens{\NumA{5}, \NumB{1}}, \Parens{\NumA{5}, \NumB{2}}, and \Parens{\NumA{6}, \NumB{1}}.\\
There are \emph{three} possible second rolls which are doubles and do immediately yield a result $\ge 12$, resulting in a partial answer of $\langle?\rangle + \frac{12 + 3 \times 36}{1506}$.\\
There are \emph{three} possible second rolls which are doubles and \emph{do not} immediately yield a result $\ge 12$ which we must consider to determine that $\langle?\rangle = \frac{26}{1506} + \frac{33}{1506} + \frac{36}{1506}$:
\begin{itemize}
\item When \Roll{2}{\Die{1} = \Die{2} = \Num{1}} then \Prob{\textbf{spaces} \ge 12} = $\frac{26}{1506}$\\
Because the first roll plus the second roll is $2 + 2 + 1 + 1 = 6$, there are \emph{ten} possible third rolls which do not yield a number of spaces $\ge 12$, notably:\\
\Parens{\NumA{1}, \NumB{1}}, \Parens{\NumA{1}, \NumB{2}}, \Parens{\NumA{1}, \NumB{3}}, \Parens{\NumA{1}, \NumB{4}},  \Parens{\NumA{2}, \NumB{1}}, \Parens{\NumA{2}, \NumB{2}}, \Parens{\NumA{2}, \NumB{3}}, \Parens{\NumA{3}, \NumB{1}}, \Parens{\NumA{3}, \NumB{2}}, and \Parens{\NumA{4}, \NumB{1}}.
\item When \Roll{2}{\Die{1} = \Die{2} = \Num{2}} then \Prob{\textbf{spaces} \ge 12} = $\frac{33}{1506}$\\
Because the first roll plus the second roll is $2 + 2 + 2 + 2 = 8$, there are \emph{three} possible third rolls which do not yield a number of spaces $\ge 12$, notably  \Parens{\NumA{1}, \NumB{1}}, \Parens{\NumA{1}, \NumB{2}} and \Parens{\NumA{2}, \NumB{2}}
\item When \Roll{2}{\Die{1} = \Die{2} = \Num{3}} then \Prob{\textbf{spaces} \ge 12} = $\frac{36}{1506}$\\
Because the first roll plus the second roll plus the minimum possible third roll is $2 + 2 + 3 + 3 + 1 + 1 \ge 12$
\end{itemize}

\item When \Roll{1}{\Die{1} = \Die{1} = \Num{1}} then \Prob{\textbf{spaces} \ge 12} = $\frac{15}{1506} + \frac{26}{1506} + \frac{33}{1506} + \frac{36}{1506} + \frac{4 + 2 \times 36}{1506} = \frac{186}{1506}$\\
Because the first roll is $1 + 1 = 2$, there are \emph{twenty-six} possible second rolls which do not result in a another roll and also do not have a cumulative sum $\ge 12$, leaving only \emph{four} which do, notably:\\
\Parens{\NumA{4}, \NumB{6}}, \Parens{\NumA{5}, \NumB{6}}, \Parens{\NumA{6}, \NumB{4}}, and \Parens{\NumA{6}, \NumB{5}}.\\
There are \emph{two} possible second rolls which are doubles and do immediately yield a result $\ge 12$, resulting in a partial answer of $\langle?\rangle + \frac{4 + 2 \times 36}{1506} $.\\
There are \emph{four} possible second rolls which are doubles and \emph{do not} immediately yield a result $\ge 12$ which we must consider to determine that $\langle?\rangle =  \frac{15}{1506} + \frac{26}{1506} + \frac{33}{1506} + \frac{36}{1506}$:
\begin{itemize}
\item When \Roll{2}{\Die{1} = \Die{2} = \Num{1}} then \Prob{\textbf{spaces} \ge 12} = $\frac{15}{1506}$\\
Because the first roll plus the second roll is $1 + 1 + 1 + 1 = 4$, there are \emph{twenty-one} possible third rolls which do not yield a number of spaces $\ge 12$, leaving only \emph{fifteen} which do, notably:\\
\Parens{\NumA{2}, \NumB{6}}, \Parens{\NumA{3}, \NumB{5}}, \Parens{\NumA{3}, \NumB{6}}, \Parens{\NumA{4}, \NumB{4}},  \Parens{\NumA{4}, \NumB{5}}, \Parens{\NumA{4}, \NumB{6}}, \Parens{\NumA{5}, \NumB{3}}, \Parens{\NumA{5}, \NumB{4}},\\ \Parens{\NumA{5}, \NumB{5}}, \Parens{\NumA{5}, \NumB{6}},\Parens{\NumA{6}, \NumB{2}}, \Parens{\NumA{6}, \NumB{3}}, \Parens{\NumA{6}, \NumB{4}}, \Parens{\NumA{6}, \NumB{5}}, and \Parens{\NumA{6}, \NumB{6}} .
\item When \Roll{2}{\Die{1} = \Die{2} = \Num{2}} then \Prob{\textbf{spaces} \ge 12} = $\frac{26}{1506}$\\
Because the first roll plus the second roll is $1 + 1 + 2 + 2 = 6$, there are \emph{ten} possible third rolls which do not yield a number of spaces $\ge 12$, notably:\\
\Parens{\NumA{1}, \NumB{1}}, \Parens{\NumA{1}, \NumB{2}}, \Parens{\NumA{1}, \NumB{3}}, \Parens{\NumA{1}, \NumB{4}},  \Parens{\NumA{2}, \NumB{1}}, \Parens{\NumA{2}, \NumB{2}}, \Parens{\NumA{2}, \NumB{3}}, \Parens{\NumA{3}, \NumB{1}}, \Parens{\NumA{3}, \NumB{2}}, and \Parens{\NumA{4}, \NumB{1}}.
\item When \Roll{2}{\Die{1} = \Die{2} = \Num{3}} then \Prob{\textbf{spaces} \ge 12} = $\frac{33}{1506}$\\
Because the first roll plus the second roll is $1 + 1 + 3 + 3 = 8$, there are \emph{three} possible third rolls which do not yield a number of spaces $\ge 12$, notably  \Parens{\NumA{1}, \NumB{1}}, \Parens{\NumA{1}, \NumB{2}} and \Parens{\NumA{2}, \NumB{2}}
\item When \Roll{2}{\Die{1} = \Die{2} = \Num{4}} then \Prob{\textbf{spaces} \ge 12} = $\frac{36}{1506}$\\
Because the first roll plus the second roll plus the minimum possible third roll is $1 + 1 + 4 + 4 + 1 + 1 \ge 12$

\end{itemize}

\end{itemize}

\pagebreak

The problem statements asks to compute $\Prob{\textbf{spaces} \ge 12}$. The hint suggests that we use the following logic:

\[
\Prob{\textbf{spaces} \ge 12} = \Prob{\textbf{spaces} \ge 12~|~\Roll{1}{\Die{1} \not= \Die{2}}} + \sum\limits_{x \in \Omega} \Prob{\textbf{spaces} \ge 12~|~ \Roll{1}{\Die{1} = \Die{2} = x}}
\]

Since $\Prob{\textbf{spaces} \ge 12~|~\Roll{1}{\Die{1} \not= \Die{2}}} = 0$ we have:

\begin{align*}
\Prob{\textbf{spaces} \ge 12}  &= \sum\limits_{x \in \Omega} \Prob{\textbf{spaces} \ge 12~|~ \Roll{1}{\Die{1} = \Die{2} = x}}\\
&= \Prob{\textbf{spaces} \ge 12~|~ \Roll{1}{\Die{1} = \Die{2} = \Num{1}}}\\
&+\,\Prob{\textbf{spaces} \ge 12~|~ \Roll{1}{\Die{1} = \Die{2} = \Num{2}}}\\
&+\,\Prob{\textbf{spaces} \ge 12~|~ \Roll{1}{\Die{1} = \Die{2} = \Num{3}}}\\
&+\,\Prob{\textbf{spaces} \ge 12~|~ \Roll{1}{\Die{1} = \Die{2} = \Num{4}}}\\
&+\,\Prob{\textbf{spaces} \ge 12~|~ \Roll{1}{\Die{1} = \Die{2} = \Num{5}}}\\
&+\,\Prob{\textbf{spaces} \ge 12~|~ \Roll{1}{\Die{1} = \Die{2} = \Num{6}}}\\
&= \frac{246}{1506} + \frac{246}{1506} + \frac{244}{1506} + \frac{235}{1506} + \frac{215}{1506} + \frac{186}{1506}\\
&=\frac{1372}{1506}
\end{align*}

We have determined that out of the 1506 possible outcomes, 1374 outcome will result in moving \emph{at least} 12 spaces!

However, each outcome does not occur with equal probability, so we cannot say  $\Prob{\textbf{spaces} \ge 12} = \frac{1374}{1506}$.\\
We will need to use conditional probabilities to determine the probability of each outcome.

\begin{align*}
\Prob{\textbf{spaces} \ge 12}  &= \sum\limits_{x \in \Omega} \Prob{\textbf{spaces} \ge 12~|~ \Roll{1}{\Die{1} = \Die{2} = x}}\\
&= \Prob{\textbf{spaces} \ge 12~|~ \Roll{1}{\Die{1} = \Die{2} = \Num{1}}}\\
&+\,\Prob{\textbf{spaces} \ge 12~|~ \Roll{1}{\Die{1} = \Die{2} = \Num{2}}}\\
&+\,\Prob{\textbf{spaces} \ge 12~|~ \Roll{1}{\Die{1} = \Die{2} = \Num{3}}}\\
&+\,\Prob{\textbf{spaces} \ge 12~|~ \Roll{1}{\Die{1} = \Die{2} = \Num{4}}}\\
&+\,\Prob{\textbf{spaces} \ge 12~|~ \Roll{1}{\Die{1} = \Die{2} = \Num{5}}}\\
&+\,\Prob{\textbf{spaces} \ge 12~|~ \Roll{1}{\Die{1} = \Die{2} = \Num{6}}}\\
&= \frac{1}{36} \times \frac{36}{36} + \frac{1}{36} \times \frac{36}{36} + \frac{1}{36} \times \frac{34}{36}\\
&+\,\frac{1}{36} \times \frac{\Parens{ 22 + 5 + \frac{33}{36} } }{36}\\
&+\,\frac{1}{36} \times \frac{\Parens{ 12 + 4 + \frac{33}{36} + \frac{26}{36} } }{36}\\
&+\,\frac{1}{36} \times \frac{\Parens{ ~\,4 + 3 + \frac{33}{36} + \frac{26}{36} + \frac{15}{36} } }{36}\\
&= \frac{36^2}{36^3} + \frac{36^2}{36^3} + \frac{34*36}{36^3} + \frac{27*36 + 33}{36^3} + \frac{16*36 + 59}{36^3} +  \frac{7*36 + 74}{36^3}\\
&=\frac{5782}{36^3}\\
&=\frac{2891}{23328}\\
&\approx 0.124
\end{align*}



% \frac{186}{1506}

\StartExtraCredit
\Problem{%
Which is the most preferable from these mutually exclusive options?\\[5mm]%
\setstretch{1.7}
\hspace*{1cm}$\tabbedLongunderstack[l]{
\,\heartsuit\,\quad\text{Pursue your passion}\\
\;\$\:\quad\text{Take the money}%
}$%
}


\end{document}
