\documentclass{ProblemSetCUNY}
\usepackage{bussproofs} \EnableBpAbbreviations
\usepackage{mathrsfs}
\usepackage{stix}
\usepackage{pgffor}
\usepackage{bm}
\usepackage{microtype}


\usepackage{tabstackengine}
\TABstackMath
\TABbinary

\AssignmentNumber{3}%
\CourseName{Probability and Statistics for Computer Science}
\CourseNumber{217}
\InstructorName{Alex Washburn}
\DueDateYear{2023}
\DueDateMonth{09}
\DueDateDay{18}


%\newenvironment{\ContactListing}[args]{begdef}{enddef}

\newcommand{\ProofLabelOne}[1]{\RightLabel{\ensuremath{\Parens{\mathrm{\textsc{#1}}}}\xspace}}
\newcommand{\ProofLabelTwo}[2]{\RightLabel{\ensuremath{\Parens{\mathrm{\textsc{#1}}}\Parens{\mathrm{#2}}}\xspace}}

\newcommand{\LogicCase}[1]{\item \textbf{Case}~#1:\newline}

\newcommand{\DerivationFormat}[2]{\hfill\qquad#2\textsc{#1}}
\newcommand{\Derivation}[2]{\DerivationFormat{#1}{#2~\ensuremath{\ruledelayed}~}}

\newcommand{\AxiomCL}[1]{\DerivationFormat{Axiom~#1}{}}
\newcommand{\Factivity}{\DerivationFormat{Factivity}{}}
\newcommand{\Given}{\DerivationFormat{Hypothesis}{}}
\newcommand{\Contrapositive}{\DerivationFormat{Contrapositive}{}}
\newcommand{\DoubleNegation}{\DerivationFormat{Double Negation}{}}
\newcommand{\DeMorgans}{\DerivationFormat{DeMorgans}{}}
\newcommand{\Necitation}[1]{\Derivation{Necitation}{#1}}
\newcommand{\Distribution}[1]{\Derivation{Distribution}{#1}}
\newcommand{\Reflexivity}[1]{\Derivation{Reflexivity}{#1}}
\newcommand{\ConDef}[1]{\Derivation{Connective Definition}{#1}}
\newcommand{\Contradiction}[1]{\ensuremath{\neg \phi \land \phi}\Derivation{Contradiction}{\ensuremath{\phi = #1}}}
\newcommand{\DedThm}[1]{\Derivation{Deduction Theorem}{#1}}
\newcommand{\ModPon}[2]{\Derivation{Modus Ponens}{#1, #2}}
\newcommand{\Syllogism}[2]{\Derivation{Syllogism}{#1, #2}}
\newcommand{\Theorem}[2]{\DerivationFormat{Theorem~--~Lecture~#1,~Page~#2}{}}

\newcommand{\R}[2]{\ensuremath{\World{#1}\mathrel{R}\World{#2}}\xspace}
\newcommand{\Tr}[2]{\ensuremath{\tau\Parens{#1,\,#2}\xspace}}
\newcommand{\T}[1]{\ensuremath{\tau\Parens{#1}\xspace}}
\newcommand{\Prob}[1]{\ensuremath{P\Parens{#1}}\xspace}
\newcommand{\ProbGiven}[2]{\ensuremath{P\Parens{#1\,|\,#2}}\xspace}
\newcommand{\Event}[1]{\ensuremath{\bm{\mathsf{#1}}}\xspace}
\newcommand{\RandVar}[1]{\ensuremath{\mathbf{#1}}\xspace}
\newcommand{\DistMap}[2]{~#1&\mapsto~~#2\\}
\newcommand{\DistCondition}[2]{~#1&\text{\normalfont iff}~~#2\\}

\newcommand{\Hint}[1]{\textit{\textls{Hint:}~#1}}

\newcommand{\One}{\ensuremath{\textbf{1}}\xspace}

\newcommand{\Heads}{\ensuremath{\mathtt{H}}\xspace}
\newcommand{\Tails}{\ensuremath{\mathtt{T}}\xspace}

\newcommand{\Die}[1]{\ensuremath{\mathbf{D}_{#1}}\xspace}
\newcommand{\Above}[1]{\ensuremath{\overline{#1}}\xspace}
\newcommand{\Below}[1]{\ensuremath{\underline{#1}}\xspace}
\newcommand{\Num}[1]{\ensuremath{\mathbb{#1}}\xspace}
\newcommand{\NumA}[1]{\ensuremath{\overline{\Num{#1}}}\xspace}
\newcommand{\NumB}[1]{\ensuremath{\underline{\Num{#1}}}\xspace}

\newcommand{\Countermodel}[4]{%
\textbf{Countermodel} $\mathcal{K}$ in #1:%
\begin{itemize}%
\item \ensuremath{W = \SetNote{\World{0}%
\foreach \n in {1,...,#2}{,~\World{\n}}%
}}%
\item \ensuremath{\mathrel{R}~=~#3}%
\item #4%
\end{itemize}%
}

\begin{document}
\CoverPage%

\newcommand{\Bullet}[3]{\hspace*{4mm}\ensuremath{\Parens{\Event{#1}}}:\hspace*{2mm}#2\[#3\]\\[5mm]}
\newcommand{\BulletLine}[3]{\hspace*{4mm}\ensuremath{\Parens{\Event{#1}}}:\hspace*{2mm}#2\hfill\ensuremath{#3}\\[5mm]}

\Problem{%
Alice and Bob each choose at random real-valued numbers $\RandVar{A}$ and $\RandVar{B}$ (respectively) from the continuous interval $\NumericRange{0}{3}$.
We assume a uniform probability law under which the probability of an event is proportional to its area.
Consider the following events:\\[5mm]
\Bullet{W}
{The magnitude of the difference of $\RandVar{A}$ and $\RandVar{B}$ is greater than $\frac{2}{3}$.}
{\left|\,\RandVar{A} - \RandVar{B}\,\right| > \frac{2}{3}}
\Bullet{X}
{At least one of the numbers is greater than $\frac{2}{3}$.}
{\RandVar{A} > \frac{2}{3}\;\lor\;\RandVar{B} > \frac{2}{3}}
\Bullet{Y}
{Alice’s number is greater than $\frac{2}{3}$.}
{\RandVar{A} > \frac{2}{3}}
\Bullet{Z}
{The two numbers are equal.}
{\RandVar{A} = \RandVar{B}}
\Hint{Draw the 2D space of $\NumericRange{0}{3} \times \NumericRange{0}{3}$ and shade the event areas.}
}

\SubProblem{Find the probability \Prob{\Event{W}}.}\vfill

\SubProblem{Find the probability \Prob{\Event{Z}}.}\vfill

\SubProblem{Find the probability \ProbGiven{\Event{X}}{\Event{Y}}.}\vfill

\SubProblem{Find the probability \ProbGiven{\Event{Y}}{\Event{X}}.}\vfill

\SubProblem{Find the probability \Prob{\Event{W} \cap \Event{Y}}.}\vfill

\SubProblem{Find the probability \ProbGiven{\Event{X}}{\Event{W}}.}\vfill

\SubProblem{Find the probability \ProbGiven{\Event{X}}{\Event{Z}}.}\vfill



\Problem{%
You have two fair \emph{four-sided} dice, one green and one blue, called \Die{1} \& \Die{2}, respectively.\\[5mm]
$\hspace*{8mm}\Die{1} \text{ has outcomes } \Omega_{\Die{1}} = \{\,\NumA{1},\,\NumA{2},\,\NumA{3},\,\NumA{4}\,\}$.\\
$\hspace*{8mm}\Die{2} \text{ has outcomes } \Omega_{\Die{2}} = \{\,\NumB{1},\,\NumB{2},\,\NumB{3},\,\NumB{4}\,\}$.\\[5mm]
Each dice roll is independent of all others, and all faces are equally likely to come out on top when the dice is rolled.
Suppose you roll the dice twice.
Consider the following events:\\[5mm]
\BulletLine{A}
{The total of two dice is \Num{8}.}
{\RandVar{\Die{1}} + \RandVar{\Die{2}} = \Num{8}}
\BulletLine{B}
{At least one die resulted in \Num{4}.}
{\RandVar{\Die{1}} = \NumA{4}\;\lor\;\RandVar{\Die{2}} = \NumB{4}}
\BulletLine{C}
{At least one die resulted in \Num{1}.}
{\RandVar{\Die{1}} = \NumA{1}\;\lor\;\RandVar{\Die{2}} = \NumB{1}}
\BulletLine{D}
{The total of two dice is \Num{5}.}
{\RandVar{\Die{1}} + \RandVar{\Die{2}} = \Num{5}}
\BulletLine{E}
{The difference between the dice is exactly \Num{1}.}
{\left|\RandVar{\Die{1}} - \RandVar{\Die{2}}\right| = \Num{1}}
\BulletLine{F}
{The green die is less than the blue die.}
{\RandVar{\Die{1}} < \RandVar{\Die{2}}}
\Hint{Draw a $4 \times 4$ grid enumerating the all possible rolls of the dice.}%
}

\SubProblem{Is event \Event{A} independent of event \Event{B}?}\vfill

\SubProblem{Is event \Event{A} independent of event \Event{C}?}\vfill

\SubProblem{Are events \Event{E} and \Event{F} independent?}\vfill

\SubProblem{Are events \Event{E} and \Event{F} independent given event D?}\vfill

\newcommand{\Coin}[1]{\ensuremath{\mathcal{C}_{\mathbf{#1}}}}

\Problem{
Consider three coins, \Coin{1}, \Coin{2}, and \Coin{3}.
Coins \Coin{1} and \Coin{2} are fair coins:
$$\ProbGiven{\Heads}{\Coin{1}} = \frac{1}{2} = \ProbGiven{\Heads}{\Coin{2}}$$
However, coin \Coin{3} is biased, yielding heads with the probability: 
$$\ProbGiven{\Heads}{\Coin{3}} =  \frac{3}{4}$$\\
You select uniformly at random one of  \Coin{1}, \Coin{2}, or \Coin{3}, and then toss the selected coin 3 times, yielding the outcome \SetNote{\Heads,\,\Heads,\,\Heads}.\\[5mm]
\textit{What is the probability you selected the biased coin \Coin{3}?}
}

\StartExtraCredit
\Problem{%
Which is the most preferable from these mutually exclusive options?\\[5mm]%
\setstretch{1.7}
\hspace*{1cm}$\tabbedLongunderstack[l]{
\,\heartsuit\,\quad\text{Pursue your passion}\\
\;\$\:\quad\text{Take the money}%
}$%
}


\end{document}
