\documentclass{ExamCUNY}
\usepackage{mathrsfs}
\usepackage{stix}
\usepackage{pgffor}
\usepackage{bm}
\usepackage{microtype}



\usepackage{tabstackengine}
\TABstackMath
\TABbinary



\ExamNumber{1}%
\CourseName{Probability and Statistics for Computer Science}
\CourseNumber{217}
\InstructorName{Alex Washburn}
\DueDateYear{2023}
\DueDateMonth{09}
\DueDateDay{28}




%\newenvironment{\ContactListing}[args]{begdef}{enddef}

\newcommand{\ProofLabelOne}[1]{\RightLabel{\ensuremath{\Parens{\mathrm{\textsc{#1}}}}\xspace}}
\newcommand{\ProofLabelTwo}[2]{\RightLabel{\ensuremath{\Parens{\mathrm{\textsc{#1}}}\Parens{\mathrm{#2}}}\xspace}}

\newcommand{\LogicCase}[1]{\item \textbf{Case}~#1:\newline}

\newcommand{\DerivationFormat}[2]{\hfill\qquad#2\textsc{#1}}
\newcommand{\Derivation}[2]{\DerivationFormat{#1}{#2~\ensuremath{\ruledelayed}~}}

\newcommand{\AxiomCL}[1]{\DerivationFormat{Axiom~#1}{}}
\newcommand{\Factivity}{\DerivationFormat{Factivity}{}}
\newcommand{\Given}{\DerivationFormat{Hypothesis}{}}
\newcommand{\Contrapositive}{\DerivationFormat{Contrapositive}{}}
\newcommand{\DoubleNegation}{\DerivationFormat{Double Negation}{}}
\newcommand{\DeMorgans}{\DerivationFormat{DeMorgans}{}}
\newcommand{\Necitation}[1]{\Derivation{Necitation}{#1}}
\newcommand{\Distribution}[1]{\Derivation{Distribution}{#1}}
\newcommand{\Reflexivity}[1]{\Derivation{Reflexivity}{#1}}
\newcommand{\ConDef}[1]{\Derivation{Connective Definition}{#1}}
\newcommand{\Contradiction}[1]{\ensuremath{\neg \phi \land \phi}\Derivation{Contradiction}{\ensuremath{\phi = #1}}}
\newcommand{\DedThm}[1]{\Derivation{Deduction Theorem}{#1}}
\newcommand{\ModPon}[2]{\Derivation{Modus Ponens}{#1, #2}}
\newcommand{\Syllogism}[2]{\Derivation{Syllogism}{#1, #2}}
\newcommand{\Theorem}[2]{\DerivationFormat{Theorem~--~Lecture~#1,~Page~#2}{}}

\newcommand{\R}[2]{\ensuremath{\World{#1}\mathrel{R}\World{#2}}\xspace}
\newcommand{\Tr}[2]{\ensuremath{\tau\Parens{#1,\,#2}\xspace}}
\newcommand{\T}[1]{\ensuremath{\tau\Parens{#1}\xspace}}
\newcommand{\Prob}[1]{\ensuremath{P\Parens{#1}}\xspace}
\newcommand{\ProbGiven}[2]{\ensuremath{P\Parens{#1\,|\,#2}}\xspace}
\newcommand{\Event}[1]{\ensuremath{\bm{\mathsf{#1}}}\xspace}
\newcommand{\RandVar}[1]{\ensuremath{\mathbf{#1}}\xspace}
\newcommand{\DistMap}[2]{~#1&\mapsto~~#2\\}
\newcommand{\DistCondition}[2]{~#1&\text{\normalfont iff}~~#2\\}

\newcommand{\Hint}[1]{\textit{\textls{Hint:}~#1}}

\newcommand{\One}{\ensuremath{\textbf{1}}\xspace}

\newcommand{\Coin}[1]{\ensuremath{\mathbf{C}_{#1}}\xspace}
\newcommand{\Heads}{\ensuremath{\mathtt{H}}\xspace}
\newcommand{\Tails}{\ensuremath{\mathtt{T}}\xspace}

\newcommand{\Die}[1]{\ensuremath{\mathbf{D}_{#1}}\xspace}
\newcommand{\Above}[1]{\ensuremath{\overline{#1}}\xspace}
\newcommand{\Below}[1]{\ensuremath{\underline{#1}}\xspace}
\newcommand{\Num}[1]{\ensuremath{\mathbb{#1}}\xspace}
\newcommand{\NumA}[1]{\ensuremath{\overline{\Num{#1}}}\xspace}
\newcommand{\NumB}[1]{\ensuremath{\underline{\Num{#1}}}\xspace}

\newcommand{\Deck}{\ensuremath{\bm{\mathcal{D}}}\xspace}
\newcommand{\CardSuits}{\ensuremath{\bm{\mathcal{S}}}\xspace}
\newcommand{\CardRanks}{\ensuremath{\bm{\mathcal{R}}}\xspace}
\newcommand{\CardRank}[1]{\ensuremath{\text{\textsf{\textbf{#1}}}}\xspace}


\newcommand{\Countermodel}[4]{%
\textbf{Countermodel} $\mathcal{K}$ in #1:%
\begin{itemize}%
\item \ensuremath{W = \SetNote{\World{0}%
\foreach \n in {1,...,#2}{,~\World{\n}}%
}}%
\item \ensuremath{\mathrel{R}~=~#3}%
\item #4%
\end{itemize}%
}

\begin{document}
\CoverPage%

\newcommand{\BulletLong}[3]{\hspace*{4mm}\ensuremath{\Parens{\Event{#1}}}:\hspace*{2mm}#2\[#3\]\\[5mm]}
\newcommand{\BulletLine}[3]{\hspace*{4mm}\ensuremath{\Parens{\Event{#1}}}:\hspace*{2mm}#2\hfill\ensuremath{#3}\\[5mm]}
\newcommand{\Bullet}[2]{\hspace*{4mm}\ensuremath{\Parens{\Event{#1}}}:\hspace*{2mm}#2\\[5mm]}

\newcommand{\True}{\ensuremath{\bm{\mathtt{True}}}\xspace}
\newcommand{\False}{\ensuremath{\bm{\mathtt{False}}}\xspace}

\newcommand{\TrueFalse}[1]{\ensuremath{\True\;\oplus\;\False\;\colon~~{#1}}}

\newcommand{\SupportFalsify}[1]{\ensuremath{\mathtt{Support}\;\oplus\;\mathtt{Falsify}\;\colon~~{#1}}}



%%%%%%%%%%%%%
% Instructions

\clearpage%                                                                                                                                                                                           
\newgeometry{bottom=20mm,left=20mm,right=20mm,top=20mm}%  
\begin{center}
{\Huge \underline{Examination Instructions}}
\end{center}
~\vspace*{5mm}
{\Large
\begin{itemize}
\item Read each question carefully, preferably twice.
\item If you do not understand a question, raise your hand and ask for clarification.
\item There are no ``trick questions.''\\
The most straight-forward interpretation of the problem statement is likely the correct interpretation.
\item Show as much of your work and thought process as possible.\\
Partial credit will be given for partially correct answers.
\item I am looking to \emph{give} credit for correctness, not \emph{deduct} credit for mistakes.\\
It is \emph{always better} to be more verbose than to be terse, since you will receive full credit as long some of your solution correctly conveys the answer.
\end{itemize}
}



\newcommand{\ExamObject}[2]{%
{\huge \underline{#1}}\\[5mm]%
{\Large #2}\\[5mm]%
}

%%%%%%%%%%%%%
% Combinatorial Objects

\clearpage%                                                                                                                                                                                           
\newgeometry{bottom=20mm,left=20mm,right=20mm,top=20mm}%  
\begin{center}
{\Huge \underline{Examination Objects Descriptions}}
\end{center}

\ExamObject{Coin}{%
All references to a ``coin'' or ``coins'' refers to the object \Coin{}. The coin \Coin{} has \emph{two} faces.
The faces are called ``heads'' and ``tails''denoted as \Heads and \Tails, respectively.
\[\Omega = \SetNote{\Heads,\,\Tails}\]
\Coin{} is a fair coin, both faces have equal chance of occurring when the coin is flipped.
}

\ExamObject{Die/Dice}{%
All references to a ``die'' or ``dice'' refers to the object \Die{}. The die \Die{} has \emph{six} faces.
\[\Omega = \SetNote{\Num{1},\,\Num{2},\,\Num{3},\,\Num{4},\,\Num{5},\,\Num{6}}\]
\Die{} is a fair die, each face has an equal chance of occurring when the die is rolled.}


\ExamObject{Deck/Card/Hand}{
All references to a ``deck'' refers to the object \Deck, which is a collection of $52$ cards.
Each ``card'' is a unique pairing of a suit and rank sets \CardSuits and \CardRanks, respectively.
\[\CardSuits = 
\SetNote{
\clubsuit,\,
\vardiamondsuit,\,
\spadesuit,\, 
\varheartsuit
}\]
\[\CardRanks = \SetNote{
\CardRank{2},\,
\CardRank{3},\,
\CardRank{4},\,
\CardRank{5},\,
\CardRank{6},\,
\CardRank{7},\,
\CardRank{8},\,
\CardRank{9},\,
\CardRank{10},\,
\CardRank{J},\,
\CardRank{Q},\,
\CardRank{K},\,
\CardRank{A}
}\]
\[\Omega = \Deck = \CardSuits \times \CardRanks = \SetNote{ (s,\,r)\;|\; s \in \CardSuits,\, r \in \CardRanks}\]
All references to a ``hand'' refer to an \emph{unordered} subset of \emph{five} cards from \Deck, without replacement; i.e. since each card is unique pair of suit and rank, a hand can never contain two cards with both the same suit and rank.\\[3mm]
The dealer is fair, each possible hand dealt from \Deck occurs with equal probability.\\[3mm]
For reference these sets have the following cardinalities:
\[ \left|\CardSuits\right| = 4,\quad\left|\CardRanks\right| = 13,\quad\left|\Deck\right| = \left|\CardSuits\right| \times \left|\CardRanks\right| = 4 \times 13 = 52 \]
The number of possible five card hands which can be dealt is equal to:
\[ \binom{52}{5} = 2,598,960 \]
}

\Problem{20}{Definitions}{%
Describe to the best of your ability the definition of each probability theory concept.
You may use the English language, mathematical notation, or some combination of both.
}

\SubProblem{5}{Independence of a Random Variable}
From probability rule sheet\vfill

\SubProblem{5}{Mutually Exclusivity of Random Variables}
From probability rule sheet\vfill

\SubProblem{5}{Collectively Exhaustive}
From probability rule sheet\vfill

\SubProblem{5}{Conditional Probability}
From probability rule sheet\vfill




\Problem{20}{True or False}{%
Decide whether each statement is \texttt{True} or \texttt{False}.\\
}

\SubProblem{4}{Alice rolls a dice \Die{} \emph{twice}.
On the first roll Alice observes the number \RandVar{X} and on the second roll Alice observes the number \RandVar{Y}.}
\TrueFalse{\Prob{\RandVar{X} + \RandVar{Y} \ge 4} = \frac{3}{4}}

$\False \implies \Prob{\RandVar{X} + \RandVar{Y} \ge 4} = \frac{33}{36} \not = \frac{27}{36}$

\SubProblem{4}{Bob rolls a dice \Die{} \emph{twice}.
On the first roll Bob observes the number \RandVar{X} and on the second roll Bob observes the number \RandVar{Y}.
Bob notices that \RandVar{Y} is one more than the previous roll \RandVar{X}.
}
\TrueFalse{\Prob{\RandVar{X} = \RandVar{Y} + 1} = \frac{1}{12}}

$\False \implies \Prob{\RandVar{X} = \RandVar{Y} + 1} = \frac{5}{36} \not = \frac{3}{36}$


\SubProblem{4}{Charlotte flips a coin \Coin{} \emph{five} times.
Charlotte observes \emph{three} \Heads and \emph{two} \Tails (ordering does not matter).
}
\TrueFalse{\Prob{3~\Heads~\textbf{and}~2~\Tails} = \frac{5}{16}}%

$\True \implies \Prob{3~\Heads~\textbf{and}~2~\Tails} = \frac{10}{2^5} = \frac{5}{16}$


\SubProblem{4}{David flips a coin \Coin{} \emph{five} times.
David observes \emph{at least two} \Tails, but forgot the outcomes of other three tosses (ordering does not matter).
}
\TrueFalse{\Prob{2~\Tails \ge 2} = \frac{1}{2}}%

$\False \implies \Prob{2~\Tails \ge 2}  = 1 - \Prob{4~\Heads} - \Prob{5~\Heads} = 1 - \frac{4}{2^5} - \frac{1}{2^5} = \frac{27}{2^5} \not = \frac{1}{2}$


\SubProblem{4}{Eve flips a coin \Coin{} and rolls a die \Die{}.
Eve observes a \Heads and that the die shows the number \RandVar{X} which is \emph{odd}.}
\TrueFalse{\Prob{\Heads \;\land\; \text{odd}(\,\RandVar{X}\,)} = \frac{3}{4}}%

$\False \implies \Prob{\Heads \;\land\; \text{odd}(\,\RandVar{X}\,)} = \Prob{\Heads} \times  \Prob{\text{odd}(\,\RandVar{X}\,)} = \frac{1}{2} \times \frac{1}{2} = \frac{1}{4} \not = \frac{3}{4}$





\Problem{20}{Support or Falsify}{%
Frank has a standard deck \Deck of $52$ playing cards. Consider each statement and either:\\[5mm]
\Bullet{Support}{State that it is $\mathtt{True}$. Explain the reason why you believe that is the case. If you feel capable of providing a proof or a sketch/outline of a proof, please do so. A proof or argument you provide does not need to be perfect to receive full credit.}
\Bullet{Falsify}{State that it is $\mathtt{False}$. Explain the reason why you believe that is the case. If you have a counterexample which shows that the statement is false, please provide it.}
In both cases you should articulate your reasoning clearly.
You will receive full credit if you convey that you have thought about the statement in the correct way and arrived at the correct conclusion.
}

\SubProblem{5}{Frank deals out \emph{5} cards from the deck $\mathcal{D}$ resulting in a distinct hand.}
\SupportFalsify{\Prob{\SetNote{
\CardRank{A}\clubsuit,\,
\CardRank{A}\vardiamondsuit,\,
\CardRank{A}\spadesuit,\,
\CardRank{A}\varheartsuit,\,
\CardRank{2}\varheartsuit
}} = \frac{1}{2,598,960}}

$\True$, It was given on the ``Description'' page of the exam.


\SubProblem{5}{The hand has \emph{four} cards of rank \RandVar{X}, and one card of rank \RandVar{Y}, where $\RandVar{X} \not = \RandVar{Y}$.\\We call this a ``Four of a kind''.}
\emph{One example of such a hand:} $\SetNote{
\CardRank{7}\clubsuit,\,
\CardRank{7}\vardiamondsuit,\,
\CardRank{7}\spadesuit,\,
\CardRank{7}\varheartsuit,\,
\CardRank{A}\varheartsuit
}$.\\
\SupportFalsify{\Prob{\text{``Four of a kind''}} = \frac{624}{2,598,960}}

$\True, 13$ choices of rank of which 4 cards will match, $48$ cards remain in \Deck for last card. $\frac{13 \times 48}{2,598,960} =  \frac{624}{2,598,960}$

\SubProblem{5}{The hand has \emph{3} cards of rank \RandVar{X}, and \emph{2} cards of ranks \RandVar{Y} and \RandVar{Z}, where $\RandVar{X} \not = \RandVar{Y} \not = \RandVar{Z}$.\\We call this a ``Three of a kind''.}
\emph{One example of such a hand:} $\SetNote{
\CardRank{3}\clubsuit,\,
\CardRank{3}\vardiamondsuit,\,
\CardRank{3}\spadesuit,\,
\CardRank{6}\spadesuit,\,
\CardRank{9}\spadesuit
}$.\\
\SupportFalsify{\Prob{\text{``Three of a kind''}} = \frac{30,576}{2,598,960}}

$\True, 13$ choices of rank of which 3 cards will match, $49$ cards in \Deck for 4\textsuperscript{th} card, $48$ cards in \Deck for 5\textsuperscript{th} card. $\frac{13 \times 48 \times 49}{2,598,960} =  \frac{30,576}{2,598,960}$

\SubProblem{5}{The hand has \emph{3} cards of rank \RandVar{X}, and \emph{2} cards of rank \RandVar{Y}, where $\RandVar{X} \not = \RandVar{Y}$.\\
We call this a ``Full House''.}
\emph{One example of such a hand:} $\SetNote{
\CardRank{3}\clubsuit,\,
\CardRank{3}\vardiamondsuit,\,
\CardRank{3}\spadesuit,\,
\CardRank{2}\spadesuit,\,
\CardRank{2}\varheartsuit
}$.\\
\SupportFalsify{\Prob{\text{``Full House''}} = \frac{30,576}{2,598,960}}

$\False, 13$ choices of rank of which 3 cards will match, $12$ remaining ranks of which 2 cards will match. \\$\frac{13 \times 12}{2,598,960} \not =  \frac{30,576}{2,598,960}$


\Problem{20}{Problem 1}{%
Gizelle rolls \emph{three} dice, \Die{1}, \Die{2}, and \Die{3} then adds the numbers together.\\
The total sum of the die faces is $14$.\\[5mm]
\emph{Find $\Prob{\Die{1} + \Die{2} + \Die{3} = 14}$}%
}

The sum of the following, mutually exclusive dice roll combinations:
\begin{itemize}
\item  $\Prob{\SetNote{\Num{6},\,\Num{6},\,\Num{2}}} \mapsto \binom{3}{1} \times \Prob{\Num{6}} \times \Prob{\Num{6}} \times \Prob{\Num{2}} = 3 \times (\frac{1}{6})^3$
\item  $\Prob{\SetNote{\Num{6},\,\Num{5},\,\Num{3}}} \mapsto \phantom{1}3! \times \Prob{\Num{6}} \times \Prob{\Num{5}} \times \Prob{\Num{3}} = 6 \times (\frac{1}{6})^3$
\item  $\Prob{\SetNote{\Num{6},\,\Num{4},\,\Num{4}}} \mapsto \binom{3}{2} \times \Prob{\Num{6}} \times \Prob{\Num{4}} \times \Prob{\Num{4}} = 3 \times (\frac{1}{6})^3$
\item  $\Prob{\SetNote{\Num{5},\,\Num{5},\,\Num{4}}} \mapsto \binom{3}{2} \times \Prob{\Num{5}} \times \Prob{\Num{5}} \times \Prob{\Num{4}} = 3 \times (\frac{1}{6})^3$
\end{itemize}
\begin{align*}
\Prob{\Die{1} + \Die{2} + \Die{3} = 14} &= 3 \times \left(\frac{1}{6}\right)^3 + 6 \times\left(\frac{1}{6}\right)^3 + 3 \times \left(\frac{1}{6}\right)^3 + 3 \times \left(\frac{1}{6}\right)^3  \\
&= \frac{3 + 6 + 3 + 3}{6^3} \\
&= \frac{15}{6^3}\\
&= \frac{5}{2} \times \frac{1}{6^2}\\
&\approx 0.069\overline{4}
\end{align*}


\Problem{20}{Problem 2}{%
Henry flips \emph{two} coins \Coin{1} and \Coin{2}.
For each \Heads they observe, they roll \emph{one} die \Die{}.
Henry knows they will roll between $0$ and $2$ dice.
However many dice Henry rolls based on the observed coin flips, they will add together the number on each die face (resulting in a sum of $0$ if no dice were rolled).\\[5mm]
Let \RandVar{X} be the sum of the die faces Henry rolls after flipping \emph{two} coins.\\[5mm]
\emph{Find $\Prob{\RandVar{X} = 4}$.}%
}
The sum of the following, mutually exclusive dice roll combinations:
\begin{itemize}
\item $\SetNote{\Tails,\,\Tails} = 1 \times 0$
\item $\SetNote{\Heads,\,\Tails} = 2 \times \Prob{\Die{1} = 4}$
\item $\SetNote{\Heads,\,\Heads} = 1 \times \Prob{\Die{1} + \Die{2} = 4}$
\end{itemize}
\begin{align*}
\Prob{\RandVar{X} = 4} &= 2 \times \Prob{\Die{1} = 4} + 1 \times \Prob{\Die{1} + \Die{2} = 4}\\
&= 2 \times \frac{6}{36} + \frac{3}{36}\\
&= \frac{12}{36} + \frac{3}{36}\\
&= \frac{15}{36}\\
&= \frac{5}{12} \approx 0.41\overline{6}\\
\end{align*}


\Problem{20}{Problem 3}{%
Isabelle discards two cards $c_1$ and $c_2$ from the deck \Deck, resulting in a smaller deck $\Deck^{\prime}$.
Isabelle observes that both discarded cards have same suit \RandVar{X}.
\[\Deck^{\prime} = \Deck \setminus \SetNote{c_1,\,c_2}\quad\text{where}\quad\textsc{Suit}(c_1) = \textsc{Suit}(c_2) = \RandVar{X} \]
\emph{Without returning the discarded cards to the deck,} Isabelle deals out a hand from $\Deck^{\prime}$.
All five cards in the dealt hand \Event{H} have the same suit \RandVar{Y}.
Note that is possible $\RandVar{X} = \RandVar{Y}$ or $\RandVar{X} \not = \RandVar{Y}$.\\[5mm]
\emph{Find the probability of dealing such a hand from the smaller deck $\Deck^{\prime}$\\ given that Isabelle has already discarded $c_1$ and $c_2$.}
\[
\ProbGiven{\textsc{Suit}(\RandVar{\Event{H}_i}) = \RandVar{Y};~\;\forall i \in \NumericRange{1}{5}}{\text{dealt from}~\Deck^{\prime}}
\]
}

There are three ways the hand could be dealt with the suit $\RandVar{X} \not = \RandVar{Y}$.

There is one way the hand could be dealt with the suit $\RandVar{X} = \RandVar{Y}$.

\begin{align*}
\ProbGiven{\textsc{Suit}(\RandVar{\Event{H}_i}) = \RandVar{Y};~\;\forall i \in \NumericRange{1}{5}}{\text{dealt from}~\Deck^{\prime}} &= \frac{1}{\binom{50}{5}} \times \Parens{3 \times \binom{13}{5} + 1 \times \binom{11}{5}}\\
&= \frac{1}{\binom{50}{5}} \times \Parens{3 \times \frac{13!}{\Parens{13 - 5}! \times 5!} + \frac{11!}{\Parens{11 - 5}! \times 5!}}\\
&= \frac{1}{\binom{50}{5}} \times \Parens{3 \times \frac{13!}{8! \times 5!} + \frac{11!}{6! \times 5!}}\\
&= \frac{1}{\binom{50}{5}} \times \Parens{3 \times \frac{13\times12\times11\times10\times9}{5!} + \frac{11\times10\times9\times8\times7}{5!}}\\
&= \frac{1}{\binom{50}{5}} \times \Parens{3 \times 13\times3\times11\times3 + 11\times2\times3\times7}\\
&= \frac{\phantom{2,11}4,323}{2,118,760}\\
\end{align*}


\end{document}
